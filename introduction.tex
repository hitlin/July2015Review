\section[1]{Introduction}

The Mu2e calorimeter{'}s purpose is to confirm that the reconstructed track of a muon-to-electron candidate is well measured and defined, and not created by any spurious hit combination within the particle tracker. In order for the calorimeter to better perform this function, faster scintillation crystals are needed. BaF$_2$ has been identified as the leading choice among the fast scintillators available. Its emission decay has two components: a 0.9 ns fast component and a slower 630 ns component. The fast BaF$_2$ emission has a peak wavelength of 220 nm, while the slow emission peaks at 300 nm. To take advantage of the fast component at 220 nm, it is necessary to suppress the slow component at 300 nm. This can be done by using solar blind photodetectors, by doping the BaF$_2$ scintillator to suppress the 300 nm light, or using optical passband filters on the photodetector that strongly attenuate near 300 nm. An appropriate photodetector will need to be large area (of the order of 1 cm$^2$) in order to accommodate the size of the scintillator. 

Solar blind large area PMTs are available, however, they are fragile, expensive and cannot be used in magnetic fields. Solid-state wide bandgap solar blind detectors, such as AlGaN and SiC APDs, are available as well, however, large area detectors have yet to be developed. RMD silicon avalanche photodiodes (APDs) have been identified as a strong photodetector candidate, since they can be fabricated with a large sensing area, $\sim$ 1 cm$^2$, are robust and compact, have little dead space, are insensitive to magnetic fields and can be made solar blind by the use of passband filters. However, RMD APDs, like many other silicon based photodetectors, have a relatively low external quantum efficiency (QE) at UV wavelengths. 

An APD is essentially a solid state replacement for a PMT and, similar to a PMT, it exhibits gain created by an impact ionization process in the device. APDs are much thinner and more compact than PMTs, and can be easily tiled with little dead space, which increases the system design flexibility. Also, APDs are insensitive to magnetic fields which could be attractive to accelerator experiments where surrounding magnetic fields prevent the use of PMTs. The gain ($\sim$ 10$^3$) of deep-diffused APDs fabricated by RMD is a factor of 10 or more higher compared to that for APDs with reach-through design fabricated by Hamamatsu and PerkinElmer. Device capacitance per unit area and excess noise factor are also lower for RMD{'}s deep-diffused APDs compared to reach-through APDs. The typical UV QE (155-300 nm) is $\sim$ 10-30\%. However, for this application, the UV response and speed of RMD{'}s APD technology needs to be enhanced. Our efforts focus on modifying fabrication of RMDs APDs with JPL's superlattice-doping process and anti-reflection coating technology to improve the APD UV QE and timing properties.
