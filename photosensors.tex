\section{Photosensors}

\subsection{Superlattice deposition}

Typical large area APDs have poor quantum efficiency in the BaF$_2$ spectral region.  However, APDs and SiPMs/MPPCs from Hamamatsu and RMD made without the normal protective epoxy coating, and therefore somewhat fragile,  have quantum efficiencies in the 200 nm region of ~17\%~\cite{sato:2013}. These devices cannot, however, discriminate between the 220 nm fast component and 300 nm slow components of BaF$_2$. The presence of the slow component limits the rate capability of the calorimeter, and can therefore be an issue in high background conditions.

Our approach is to transform a large-area (9$\times$9 mm$^2$) RMD APD~\cite{RMD}, which has high gain (up to 2000) and low capacitance 0.7pf/mm$^2$, into a superlattice-doped APD~\cite{hoenk:2013} and to incorporate an atomic layer deposition (ALD) antireflection filter~\cite{hennessy:2015} that provides ~60\% quantum efficiency at 220 nm and ~0.1\% efficiency at 300 nm, thereby enabling us to obtain a larger number of photoelectrons/MeV, and also to take full advantage of the fast decay time component of BaF$_2$.

Superlattice (2D) doping, a JPL-developed surface passivation technique, produces stable surfaces on silicon photosensors. These techniques were developed to overcome surface damage due to ultraviolet radiation in satellite instrumentation. The subsurface structures are formed using a combination of molecular beam epitaxy and controlled crystalline silicon growth. After growing a thin layer of undoped silicon, a monolayer of boron is deposited, then another silicon layer is grown, and the process is repeated up to four times. The resulting subsurface structure, with high quality self-organized layers of boron atoms at densities of $\sim\!\! 2\times 10^{14}/{\rm cm}^2$ (see Figure~\ref{fig:delta}),

\begin{figure}[h!]
\centering
\includegraphics[width=0.9\linewidth]{Figures/delta.png}
\caption{A four-layer superlattice grown by MBE, in
which four 2D-doped layers are separated by 1 nm. This 2D doping creates a maximum field of 10$^7$ V/cm. The 2D doping superlattice forms a surface
depletion layer with a fixed width of less than 1 nm, which is stable against interface trap densities in excess of
10$^{14}$ cm$^{-3}$, enabling the detector to remain stable even when the surface is severely damaged by irradiation by DUV radiation. }
\label{fig:delta}
\end{figure}
\begin{figure}[h!]
\centering
\includegraphics[width=0.9\linewidth]{Figures/timing.png}
\caption{Response of unmodified and superlattice-doped 9$\times$9 mm$^2$ RMD APDs. }
\label{fig:timing}
\end{figure}

The reason for the low QE achieved by most solid state devices in the UV region is two-fold: in typical devices there is an undepleted region of tens of microns between the Si/SiO$_2$ passivation region on the absorbing surface and the depletion region that is in close proximity to the avalanche region.
Due to quantum exclusion, the superlattice structure suppresses recombination of charge at the surface, thereby improving the QE in the 220 nm region to close to the theoretical maximum. The greatly reduced undepleted region of the superlattice-doped device also produces substantially improved timing characteristics (see Figure~\ref{fig:timing}). Second, most devices have a protective resin layer that lowers the net quantum efficiency. This layer is not required in our device, which is passivated and protected by the creation of an atomic layer deposition filter described below.

\subsection{Atomic layer deposition filter}

It then remains to apply a multilayered atomic layer deposition (ALD) coating to serve as a bandwidth-reducing interference and antireflection filter.  These films, in our case alternating layers of aluminum and aluminum oxide, are created by forming a series of single atomic layers through self-limiting chemical reactions with the substrate. This process produces uniform pinhole-free layers of precise thickness. Figure~\ref{fig:filter} shows the calculated transmission of such a filter as a function of wavelength. For a five-layer ALD coating, the QE at normal incidence for the fast component of BaF$_2$ is close to 70\%, and the extinction at the slow component wavelength is nearly complete. The transmission of the interference filter is a function of incident angle, which is shown in Figure~\ref{fig:angle}.



\begin{figure}[h!]
\centering
\includegraphics[width=0.9\linewidth]{Figures/filter1.png}
\caption{Quantum efficiency (QE) as a function of wavelength for  three, five seven layer ALD filters. }
\label{fig:filter}
\end{figure}

\begin{figure}[h!]
\centering
\includegraphics[width=0.9\linewidth]{Figures/angle.png}
\caption{Angular response of the five layer ALD filter as a function of wavelength.}
\label{fig:angle}
\end{figure}


\subsection{Electrical performance}

We have produced several versions of superlattice/ALD-modified RMD APDs thus far. The anticipated filter response and quantum efficiency improvements have been experimentally demonstrated. The devices, however, have somewhat elevated dark current and noise, as seen in Figure~\ref{fig:noise}. Operation at somewhat reduced temperatures improves the noise performance substantially: at $\sim\!\! 25^\circ$ below ambient, the noise of the superlattice/ALD-modified devices is equivalent to that of standard devices at room temperature.

Several process variations are being investigated to reduce the dark current and associated noise. These include variation of the parameters of the superlattice deposition layers, modification of the first dielectric layer of the ALD filter and changes to the electrical contact geometry. The current devices are nonetheless suitable for deployment in the Mu2e calorimeter.

\begin{figure}[h!]
\centering
\includegraphics[width=0.9\linewidth]{Figures/noise.png}
\caption{Noise in {\t rms} electrons of a standard and a superlattice-doped RMD APD as a function of gain.}
\label{fig:noise}
\end{figure}

\subsection{Device Characterization}
We characterize various types of the standard and superlattice (SL) RMD APD's at different temperatures, which involves taking both constant illumination and pulsed illumination measurements.  These measurements allow us to characterize the dark current, gain, and noise of the devices.  For the constant illumination measurements, the current output by the photosensor is recorded with and without an LED of appropriate wavelength incident on the surface, and is used to calculate the gain.  For the pulsed illumination measurements, the LED is driven by a waveform generator and the response of the device, a pulse, is digitized and recorded by an MCA, which is then stored as an energy spectrum.  The position and the width of the peak in the spectrum are used to quantify the gain and noise. 

First we present the constant illumination measurements.  First device characterized is a 9x9mm standard RMD APD, which serves as a baseline reference in terms of amount of dark current, maximum gain, and noise.  Fig.~\ref{fig:stdconst} shows for various temperatures a) dark current as a function of the bias voltage, b) gain as a function of the bias voltage, and c) dark current as a function of the gain.

Similarly, Fig~\ref{fig:slconst} shows the same measurements as Fig.1 but for a 9x9mm SL-APD with a 5-layer filter. As we can see, this particular device does not achieve a gain of 500 above $0^{\circ}$C, which is the nominal operating gain that we aim for.  We can compare the performances of the two device by comparing the dark current at gain 500 as shown in Fig.3.  We see that the SL-APD needs to be cooled down to $-10^{\circ}$C before the dark current becomes comparable to that of the standard device at room temperature ($25^{\circ}$C).

The pulsed measurments allow us to quantify the noise and the gain, which can be combined to determine the best operating bias voltage where the signal-to-noise ratio is maximized.  

First, the device response to the pulsing LED is recorded.  Th width of the resulting gaussian peak has contributions from the electronic noise and the excess noise of the device.  Next, we inject a tail pulse with $110\mu s$ decay time into the test input of the charge-sensitive preamplifier, which allows us to directly measure the electronic noise.  Lastly, we expose the photosensor to a $^{55}$Fe source which emits 5.9 keV x-rays for energy calibration purposes.  The noise reported in this section is in units of eV(Si), or eV in silicon, which has a conversion factor of 3.62 eV(Si)/electron.  

Put pulsed measurement plots here.

\begin{figure}
  \centering
  \begin{subfigure}[h]{0.9\linewidth}
    \includegraphics[width=\linewidth]{Figures/StdGainCurves.png}
    \caption{Gain curves for 9x9mm standard RMD APD at various temperatures}
  \end{subfigure}

  \begin{subfigure}[h]{0.9\linewidth}
    \includegraphics[width=\linewidth]{Figures/StdDarkCurrentVsGain.png}
    \caption{Dark current for 9x9mm standard RMD APD at various temperatures}
  \end{subfigure}
  \label{fig:stdconst}
  \caption{Constant illumination measurements for 9x9mm standard device}
\end{figure}

\begin{figure}
  \centering
  \begin{subfigure}[h]{0.9\linewidth}
    \includegraphics[width=\linewidth]{Figures/SLGainCurves.png}
    \caption{Gain curves for 9x9mm SL APD at various temperatures}
  \end{subfigure}

  \begin{subfigure}[h]{0.9\linewidth}
    \includegraphics[width=\linewidth]{Figures/SLDarkCurrentVsGain.png}
    \caption{Dark current for 9x9mm SL APD at various temperatures}
  \end{subfigure}
  \label{fig:slconst}
  \caption{Constant illumination measurements for 9x9mm SL device}
\end{figure}

\subsection{Radiation Hardness of APD's}

We also test the radiation hardness of both the standard and SL APD's when exposed to $\gamma$ and neutron radiation.  We prepare two sets of devices, each set consisting of one 9x9mm standard APD and one 9x9 SL APD, then expose them to 1, 10, then 100 krads of $\gamma$ radiation or $10^9$, then $10^{10}$ neutrons/cm$^2$ of neutron radiation.  After exposure to each magnitude, the devices were re-characterized to quantify any degradation.  

\subsection{Detecting BaF$_2$ Scintillation}

We couple BaF$_2$ crystals of two different form factors, 10x10x13mm and 30x30x250mm, to the photosensors with optical grease and shine various radioactive sources and record the energy spectra.  The larger crystal size is close to the actual crystal size to be placed in the calorimeter.  The noise of the SL devices was too high at room temperature to see the BaF$_2$ scintillation light from a $^{137}$Cs source. As comparison, we measure both the standard and SL devices at $-15^{\circ}$; the respective spectra are shown in Fig. 4.  

Put BaF2 spectra here.

As we can see, the standard device achieved an excellent energy resolution of 4.6\%.  While we can see some evidence for the 662 keV photopeak above the noise in the SL device, the finer structures such as the compton edge and the back-scatter peak could not be discerned.  

\begin{figure}[h!]
\begin{center}
\includegraphics[width=0.7\columnwidth]{Figures/RMD_process.png}
\caption{Replace this text with your caption.}
\end{center}
\end{figure}

\begin{figure}[h!]
\begin{center}
\includegraphics[width=0.7\columnwidth]{Figures/fivelayer.png}
\caption{Calculated response of five layer ALD filter.}
\end{center}
\end{figure}
