\section{Energy calibration}
%disable \color
\def\color#1{\relax}
\def\textcolor#1#2{#2}

The calorimeter requirements relevant to energy calibration are:

\begin{itemize}
\item Provide online calibration sufficient for calorimeter trigger.
\item Precision commensurate with calorimeter resolution requirement of $\sigma_E\equiv \hbox{FWHM}/2.35 \sim 5$\% at 100 MeV. %(864 says rms)
\item Absolute precision and stability better than 1\%.
\begin{itemize}
\item Absolute scale independent of tracker.
\end{itemize}
\item Because the crystals are different, an independent calibration is required for each crystal. 
\item Light output is time-dependent, hence the time-dependence must be tracked (over a scale of weeks).
\item Perform a (source) calibration of entire calorimeter in $\sim10$ minutes.
\end{itemize}

The calibration tools that we have available include:

\begin{enumerate}
\item Electronics pulser (FEE)
\item Temperature monitoring
 \item Light flasher system %docdb-2624 (LYSO), TDR
 \item  {\color{red} Source calibration system, 6.13 MeV} %docdb-2904
 \item {\color{magenta} Decays in orbit} %docdb-4618, docdb-4614(tracker)
\begin{itemize}
\item Compare with tracker 
\item Different $B$ fields for different momentum ranges 
\item Absolute spectrum well known at mid-energies 
\end{itemize}
 \item {\color{blue} Cosmic rays} %docdb-2991, docdb-3052
 \item {\color{brown} 70 MeV $e^+$ from $\pi^+\to e^+\nu_e$} 
\end{enumerate}

The 6 MeV source calibration system produces 6 MeV photons from excited oxygen-16
produced via neutron activation of fluorine in the chain:
\begin{eqnarray}
^{19\!}F + n \to\! &^{16}N& +\ \alpha\cr
 &^{16}N& \to\ ^{16\!}O^* + \beta\quad \textcolor{red}{t_{1/2} = 7\hbox{ s}}\cr
  &\phantom{^{16}N}& \phantom{\to}\ \ ^{16\!}O^* \to ^{16\!\!\!}O + \gamma (6.13\hbox{ MeV})\nonumber
\end{eqnarray}
The source is in a bunker off-detector, where the neutrons are produced with a DT generator
with approximately 14 MeV kinetic energy. Fluorinert (3M's FC-770) provides the fluorine. It is activated
in a bath around the DT generator, and the fluid is pumped to thin-wall aluminum tubing at the
faces of the two calorimeter disks.  

The basic strategy for calibration is as follows:
The initial crystal-by-crystal calibration is performed using the 6 MeV source. We plan to do this
pre-installation, and then of order once every week during operation of the experiment. This
calibration will serve as the online calibration, in the calorimeter trigger.

Higher energy with electrons from muons decaying in orbit (DIOs)
 Interpolation and extrapolation with source
 [Tracker could be used] 
 
Absolute spectrum (at lower fields)
 Check of MC extrapolation

Monitor electronic gains with pulser
 Monitor APD temperature
 Monitor crystal transmission with light pulser

 Investigate maintaining detailed crystal-by-crystal time-dependent model in MC
using calibration and monitoring data

Cosmic rays as independent check

$\pi^+\to e^+\nu_e$ as independent check
